%-*-coding: utf-8-*-

\chapter{Описание предложенного алгоритма}
В данной главе будет описан предложенный алгоритм консенсуса на основе комитета участников, проведен его анализ и доказательство,
а также возможные модификации.

\section{Схема предложенного алгоритма}
В данном разделе будет описана основная схема и краеугольные камни алгоритма.

В основе алгоритма находится комитет участников сети, 
которые принимают решения по всем жизненно важным процесам криптовалюты, таким как: создание новых блоков с транзакциями, их верификацию и другие. 
Каждый участник может быть идентифицируем некоторым образом, например, его публичным ключом, и каждый из участников
знает индефикаторы всех остальных участников комитета.
Комитет состоит из $n$ участников сети, размер его не меняется с течением времени. 
Новый участник может присоединиться к нему, при этом один из существующих должен покинуть его, чтобы размер оставался равным $n$.

Комитет ответственнен за корректное состояние хранилища и порядка операций, изменяющих его.
Если какой-то из участников хочет внести некоторое изменение в хранилище, остальные участники должны прийти к консенсусу, является ли оно корректным. 
Выбор алгоритма, с помощью которого участники будут приходить к общему решению, является одной из важных частей решения.  
По сути, участники должны реализовывать SMR(State Machine Replication)[ссылка].

Таким образом, алгоритм условно можно разбить на две составляющие: алгоритм SMR в его основе и то, каким образом новые участники попадают в комитет.

TODO картинка

\section{Свойства SMR}
В данном разделе мы обозначим свойства, которые должен предоставлять SMR алгоритм.
В в следующих разделах рассмотрим возможных канидатов, удовлетворяющих этому свойству.

Первое из таких свойств \textit{консистентность}[ссылка на Hybrid consensus], включает в себя:
\begin{itemize}
\item Общий префикс - TODO
\item Самоконсистентность - TODO
\end{itemize}