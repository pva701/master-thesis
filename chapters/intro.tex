\startprefacepage
В последние годы криптовалюты набирали популярность как среди рядовых пользователей как альтернатива фиатным способам совершения платежей, а также как плодородная почва для исследования.
Большинство криптовалют реализованы таким образом, чтобы предоставить конечному пользователю ряд свойств и гарантий, которыми не обладают существующие фиатные финансовые институты. К таким свойствам относятся децентрализация, анонимность, безопаность и масштабируемость.
 
Достижение этих свойств одновремено зачастую является непростой задачей, 
поэтому они достигаются не в полной степени или с некотороми компромиссами.
Например, для достижения безопасности используются алгоритмы на основе доказательства работы (Proof of Work)\cite{pow}
что зачастую вызывает другие недостатки, такие как уменьшение скорости обработки транзакций или уязвимость к некоторым типам атак.
Другие же криптовалюты используют для обеспечения безопасности алгоритмы на основе доказательства доли владения (Proof of Stake)\cite{pos}, в которых 
возникают другие проблемы, такие как проблема "ничего на кону" (nothing-at-stake)\cite{pos},
что приводит к централизации криптовалюты.

%Актуальность
Одной из таких проблем современных криптовалют является скорость обработки транзакций. 
Например, Bitcoin способен обрабатывать лишь 7 транзакций в секунду. Данный недостаток вытекает из того,
что многие криптовалюты предоставляют только \textit{консистентность в конечном счете}\cite{DBLP:journals/corr/DeckerSW14}, то есть транзакции попавшие в блокчейн могут быть отменены, но утверждается, что в конечно счете, они окажутся в блокчейне. Данная гарантия была достаточной на рассвете зарождения технологии,  но является неудовлетворительной для текущих требований.

%Новизна
В данной работе будет предложен алгоритм консенсуса, который предоставляет строгии гарантии безопасности, 
вместе с тем высокую скорость обработки транзакций, в то же время оставаясь децентрализованным.
Преимуществом данного алгоритма также является то, что в нем не может быть откактов блоков,
что дает дополнительную гарантию, что однажды попавшая в блокчейн транзакция не может быть отменена.
Также в работе будет рассмотрено применение данного алгоритма как в сочетании с системой на основе доказательства выполненной работы,
так и с системой на основе доказательства долей владения.

% Структура работы
В главе 1 будут введены вспомогательные понятия, сформулированы решаемые задачи.
Также будет проведен обзор предметной области, существующие решения их недостатки и проблемы.

В главе 2 будут рассмотрены предложенные улучшения и их вариации, принцип работы и обоснование.

В главе 3 будет проведен анализ алгоритма, рассмотрены возможные атаки и показано, как алгоритм будет справляться с ними.
Также будут предоставлены результаты замеров скорости работы и потребляемой памяти.