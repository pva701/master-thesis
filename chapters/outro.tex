В рамках данной работы был предложен алгоритм консенсуса для распределенных геореплицируемых систем.
Два основных преимущества предложенного решения по сравнению с существующими состоят в том, что в алгоритме не может быть вилок, таким образом он обладает строгой консистентностью, второе, но не менее важное преимущество~--- уменьшены затраты на коммуникацию между участниками системы.

Алгоритм был рассмотрен в контексте разработки криптовалют, подробно описан в главе 2, а также проанализирован в главе 3. В результате анализа было доказано, что алгоритм обладает консистентностью и обеспечивает прогресс системы, также были приведены оценки на ожидаемое время обработки транзакций. Тем самым были продемонстрировано, что алгоритм предоставляет необходимые гарантии и показана жизнеспособность.

Основным недостатком предложенного решения является то, что в его основе лежит подход на основе доказательства проделанной работы, который требует использования большого количество энергии. Другим же недостатком является требование на нахождение в сети как минимум $2f+1$ активных участников. Данные проблемы могут быть решены, используя другие подходы, такие как доказательство доли владения и другие, которые лишены данных проблем, хотя и обладают другими.

В заключении, хочется отметить, что предложенный в данной работе подход~--- это еще один шаг в сторону более устойчивых и быстрых алгоритмов консенсуса. Алгоритм не лишен недостатков, но многие из них обозначены в рамках работы и могут быть устранены во время дальнейшей его разработки.

% Среднее время нахождения в мемпуле
% Среднее время за которое коммитится слот
% Как минимум f+1 имею одинаковый слот