\documentclass[specification,annotation,times]{itmo-student-thesis}
%\documentclass[specification,annotation]{itmo-student-thesis}

%% Опции пакета:
%% - specification - если есть, генерируется задание, иначе не генерируется
%% - annotation - если есть, генерируется аннотация, иначе не генерируется
%% - times - делает все шрифтом Times New Roman, собирается с помощью xelatex
%% - pscyr - делает все шрифтом Times New Roman, требует пакета pscyr.

%% Делает запятую в формулах более интеллектуальной, например:
%% $1,5x$ будет читаться как полтора икса, а не один запятая пять иксов.
%% Однако если написать $1, 5x$, то все будет как прежде.
%\usepackage{icomma}

%% Один из пакетов, позволяющий делать таблицы на всю ширину текста.
\usepackage{tabularx}

%% Данные пакеты необязательны к использованию в бакалаврских/магистерских
%% Они нужны для иллюстративных целей
%% Начало
\usepackage{tikz}
\usetikzlibrary{arrows}
\usepackage{filecontents}
\usepackage{mathtools}
\DeclarePairedDelimiter{\ceil}{\lceil}{\rceil}

\begin{filecontents}{master-thesis.bib}
@incollection{ nsga-ii-steady-state,
    year        = {2009},
    booktitle   = {Nature-Inspired Algorithms for Optimisation},
    number      = {193},
    series      = {Studies in Computational Intelligence},
    title       = {On the Effect of Applying a Steady-State Selection Scheme in the Multi-Objective Genetic Algorithm {NSGA}-{II}},
    publisher   = {Springer Berlin Heidelberg},
    author      = {Nebro, Antonio J. and Durillo, Juan J.},
    pages       = {435-456},
    langid      = {english}
}


@inproceedings{ example-english,
    year        = {2015},
    booktitle   = {Proceedings of IEEE Congress on Evolutionary Computation},
    author      = {Maxim Buzdalov and Anatoly Shalyto},
    title       = {Hard Test Generation for Augmenting Path Maximum Flow 
                   Algorithms using Genetic Algorithms: Revisited},
    pages       = {2121-2128},
    langid      = {english}
}

@article{ example-russian,
    author      = {Максим Викторович Буздалов},
    title       = {Генерация тестов для олимпиадных задач по программированию 
                   с использованием генетических алгоритмов},
    journal     = {Научно-технический вестник {СПбГУ} {ИТМО}},
    number      = {2(72)},
    year        = {2011},
    pages       = {72-77},
    langid      = {russian}
}

@article{ unrestricted-jump-evco,
    author      = {Maxim Buzdalov and Benjamin Doerr and Mikhail Kever},
    title       = {The Unrestricted Black-Box Complexity of Jump Functions},
    journal     = {Evolutionary Computation},
    year        = {2016},
    note        = {Accepted for publication},
    langid      = {english}
}

@book{ bellman,
    author      = {R. E. Bellman},
    title       = {Dynamic Programming},
    address     = {Princeton, NJ},
    publisher   = {Princeton University Press},
    numpages    = {342},
    pagetotal   = {342},
    year        = {1957},
    langid      = {english}
}
\end{filecontents}

%% Указываем файл с библиографией.
\addbibresource{master-thesis.bib}

\begin{document}

\studygroup{M4238}
\title{Разработка алгоритма консенсуса на основе комитета участников}
\author{Пересадин Илья Валерьевич}{Пересадин И.В.}
\supervisor{Шалыто Анатолий Абрамович}{Шалыто А.А.}{проф., д.т.н.}{главный научный сотрудник Университета ИТМО}
\publishyear{2019}
%% Дата выдачи задания. Можно не указывать, тогда надо будет заполнить от руки.
\startdate{01}{сентября}{2018}
%% Срок сдачи студентом работы. Можно не указывать, тогда надо будет заполнить от руки.
\finishdate{31}{мая}{2019}
%% Дата защиты. Можно не указывать, тогда надо будет заполнить от руки.
\defencedate{15}{июня}{2019}

\secretary{Павлова О.Н.}

%% Задание
%%% Техническое задание и исходные данные к работе
\technicalspec{В рамках данной работы требуется разработать и доказать алгоритм на основе комитета участников, который способен работать в инклюзивной модели.
Алгоритм должен решать проблемы присущие существующим аналогичным алгоритмам, быть эффективным и предоставлять высокии гарантии безопасности.
Требуется изучить последние достижения в этой области, проанализировать их проблемы и возможные пути решения. 
 }

%%% Содержание выпускной квалификационной работы (перечень подлежащих разработке вопросов)
\plannedcontents{
\begin{enumerate}
    \item Постановка задачи и обзор предметной области
    \item Описание предложенного алгоритма
    \item Анализ предложенного алгоритма
\end{enumerate}
}

%%% Исходные материалы и пособия 
\plannedsources{
\begin{enumerate}
    \item Miguel Castro, Practical Byzantine Fault Tolerance;
    \item Leslie Lamport, The Byzantine Generals Problem.
\end{enumerate}
}

%%% Цель исследования
\researchaim{Разработать эффективный и безопасный инклюзивный алгоритм консенсуса на основе комитета участников.} 

%%% Задачи, решаемые в ВКР
\researchtargets{
  \begin{enumerate}
    \item анализ алгоритмов консенсуса в существующих криптовалютах, их проблем и недостатков;
    \item описание алгоритма не имеющих данных проблем;
    \item доказательство описанного алгоритма;
    \item анализ устойчивости предложенного алгоритма к атакам.
  \end{enumerate}
}

%%% Использование современных пакетов компьютерных программ и технологий
\addadvancedsoftware{Использовалась система контроля версий Git и система компьютерной верстки LaTeX.}{\ref{chapter1}, \ref{chapter2}, \ref{chapter3}}

%%% Краткая характеристика полученных результатов 
\researchsummary{Разработан, доказан и проанализирован инклюзивный алгоритм консенсуса на основе комитета участников, который превосходит существующие по некоторым показателям. }

\researchfunding{
    Грантов или других форм государственной поддержи и субсидирования в процессе работы не предусматривалось.
}
 
\researchpublications{Работа была опубликована в журнале и в сборнике конференции
\begin{refsection}
\nocite{article-peresadin, conference-peresadin}
\printannobibliography
\end{refsection}
}

%% Эта команда генерирует титульный лист и аннотацию.
\maketitle{Магистр}

%% Оглавление
\tableofcontents

% Chapters
\startprefacepage
В последние годы криптовалюты набирали популярность как среди рядовых пользователей как альтернатива фиатным способам совершения платежей, а также как плодородная почва для исследования.
Большинство криптовалют реализованы таким образом, чтобы предоставить конечному пользователю ряд свойств и гарантий, которыми не обладают существующие фиатные финансовые институты. К таким свойствам относятся децентрализация, анонимность, безопаность и масштабируемость.
 
Достижение этих свойств одновремено зачастую является непростой задачей, 
поэтому они достигаются не в полной степени или с некотороми компромиссами.
Например, для достижения безопасности используются алгоритмы на основе доказательства работы (Proof of Work)\cite{pow}
что зачастую вызывает другие недостатки, такие как уменьшение скорости обработки транзакций или уязвимость к некоторым типам атак.
Другие же криптовалюты используют для обеспечения безопасности алгоритмы на основе доказательства доли владения (Proof of Stake)\cite{pos}, в которых 
возникают другие проблемы, такие как проблема "ничего на кону" (nothing-at-stake)\cite{pos},
что приводит к централизации криптовалюты.

%Актуальность
Одной из таких проблем современных криптовалют является скорость обработки транзакций. 
Например, Bitcoin способен обрабатывать лишь 7 транзакций в секунду. Данный недостаток вытекает из того,
что многие криптовалюты предоставляют только \textit{консистентность в конечном счете}\cite{DBLP:journals/corr/DeckerSW14}, то есть транзакции попавшие в блокчейн могут быть отменены, но утверждается, что в конечно счете, они окажутся в блокчейне. Данная гарантия была достаточной на рассвете зарождения технологии,  но является неудовлетворительной для текущих требований.

%Новизна
В данной работе будет предложен алгоритм консенсуса, который предоставляет строгии гарантии безопасности, 
вместе с тем высокую скорость обработки транзакций, в то же время оставаясь децентрализованным.
Преимуществом данного алгоритма также является то, что в нем не может быть откактов блоков,
что дает дополнительную гарантию, что однажды попавшая в блокчейн транзакция не может быть отменена.
Также в работе будет рассмотрено применение данного алгоритма как в сочетании с системой на основе доказательства выполненной работы,
так и с системой на основе доказательства долей владения.

% Структура работы
В главе 1 будут введены вспомогательные понятия, сформулированы решаемые задачи.
Также будет проведен обзор предметной области, существующие решения их недостатки и проблемы.

В главе 2 будут рассмотрены предложенные улучшения и их вариации, принцип работы и обоснование.

В главе 3 будет проведен анализ алгоритма, рассмотрены возможные атаки и показано, как алгоритм будет справляться с ними.
Также будут предоставлены результаты замеров скорости работы и потребляемой памяти.

%-*-coding: utf-8-*-

\graphicspath{ {images/} }

\startrelatedwork

\chapter{Обзор предметной области} \label{chapter1}
В данной главе проводится обзор предметной области.
Дается объяснение технологии <<блокчейн>>, понятия алгоритма консенсуса, транзакциях и
криптографических примитивов.
Далее следует рассмотрение существующих решений и описание их недостатков.

\section{Технология блокчейн}

Технология блокчейн, от английского blockchain, дословно переводится  как "цепочка блоков".

Блок хранит в себе данные, а также ссылку на предыдущий блок. 
Блоки образуют бесконечную последовательность, которая имеет начало, но не имеет конца.
Первый блок в блокчейне называется \textit{генезис блоком} (genesis block).

По сути блокченй представляет односвязный список, где каждый элемент знает ссылку на предыдущий. 
Однако, особенностью данного списка является то,  что в качестве ссылки на предыдущий блок 
используется хэш криптографически стойкой хэш-функции содержимого предыдущего блока, 
которое включает как его данные, так и ссылку на предыдущий блок. 
Поэтому даже при малейшей попытке заменить содержимое блока, 
поменяется значение хэш-функции его данных и ссылка на него от последующего блока будет недействительна.
Нахождение двух блоков с разным содержанием и одинаковым значением хэш-функции является  задачей, требующей экспоненциального количества вычислений. 
Таким образом, хэш последнего блока в цепочке является доказательством данной цепочки, то есть легко проверить что данный хэш соответствует цепочке, но создать другую цепочку с таким же хэшом является вычислительно сложной задачей. Данная особенность ключевая, и  является одной из основых для обеспечения безопасности в криптовалютах использующих технологию блокчейн.

\begin{figure}[h]
\includegraphics[scale=0.6]{Blockchain_Scheme}
\caption{\textbf{Схема блокчейна}}
\label{fig:blockchain}
\end{figure}

На Рисунке \ref{fig:blockchain} обозначено схематиченое представление блокчейна.

Будем называть описанную структуру ~--- блокчейн.

\section{Алгоритм достижения консенсуса}
Исследование в области распределенных систем началось задолго до  появление криптовалют, и насчитывает 30 лет исследования и разработки.
Алгоритмы достижения консенсуса между несколькими участниками ~--- являются одной из популярных задач в этой области.
Формально, класс таких алгоритмов можно описать следующим образом:
\par \textit{
Имеется $n$ участников, между которыми установлены каналы связи. Каждый участник знает некоторое число $x$ (у каждого свое).
После некоторой коммуникации по каналам связи они хотят получить некоторое число $c$ (общее для всех),  равное одному из чисел $x$.
}

В рамках данной работы нас будут интересовать алгоритмы, которые предоставляют \textit{отказоустойчивость к Византийским ошибкам} (BFT)\cite{Lamport:1982}.
Византийская ошибка ~--- это ошибка при которой участник может отправлять любые данные по каналу связи, в том числе участник может препятствовать достижению консенсуса.
Впервые задача BFT и ее решения были описаны в работе Лампорта от 1982 года\cite{Lamport:1982}. 
В данной статье Лампорт также доказал, что не существует алгоритма, который приводит менее $3f+1$ участников в консенсусу, если среди них имеется $f$  византийских участников.
Иными словами, в алгоритме в котором участвуют $n$ участников менее $\ceil{\frac{n}{3}}$ могут быть византийскими.

Несмотря на то, что работа Лампорта имела важное теоретическое значение, предложенные в ней алгоритмы требовали экспоненциального количества сообщений между участниками, 
и едва ли могли применяться на практике для больших $n$. Поэтому в 1999 году был разработан алгоритм PBFT (Practical BFT)\cite{pbft},  в рамках которого отправляется $O(n^2)$  сообщений.

Хотя после разработки PBFT было предложено множество его улучшений \cite{qu, hq, Zyzzyva}, все они предполагали \textit{эксклюзивную} (permissioned) модель сети, 
в которой участники были заранее известны и не менялись со временем. Другая альтернатива ~--- \textit{инклюзивная} (permissionless) модель, в которой участники
неизвестны заранее и могут меняться со временем. Инклюзивная модель в большей степени подходит для консенсус алгоритмов в криптовалютах. Одним из самых известных инклюзивных алгоритмов является Nakamoto consensus\cite{nakamoto}, на основе которого построена криптовалюта Bitcoin. Nakamoto consensus и некоторые другие инклюзивные алгоритмы будут рассмотрены подробнее в следующих разделах данной главы.

\section{Транзакции и хранилище} \label{sec:tx}
Основной функционал в криптовалюте ~--- это возможность переводить другим участникам системы средства, а также получать средства на свой счет.
Рассмотрим один из способов реализации данного функционала,
в том числе, как хранится информация о счетах и как производится перевод средств между ними.

У каждого участника системы имеется \textit{адрес}, на котором хранятся его \textit{токены}, токены
являются аналогом фиатной валюты, например рубля или доллара. Количество токенов участника на его адресе будем называть \textit{балансом}. Токены между адресами пересылаются с помощью \textit{транзакции}. Транзакция ~--- это подтвержденное криптографической подписью сообщение, которое указывает с каких и на какие адреса отправлять и получать токены, а также количество токенов.

Вся информация о балансах хранится в базе данных, которую в рамках данной работы носит название \textit{хранилищем}.
Данная база данных может быть распределенной, тогда части ее будут храниться на устройствах участников системы. 
Такой подход называется шардирование (от английского sharding). 
Другой, более простой подход, состоит в том, что копия этой базы данных хранится у каждого участника системы.
В данной работе будет использоваться данный более простой подход.

Обозначим через $\boldsymbol{\sigma}$ отображение из адреса в состояние адреса. Тогда для адреса $a$ в $\boldsymbol{\sigma}[a]$ содержится:
\begin{itemize}
\item balance ~--- целая величина, количество токенов принадлежащих адресу $a$
\item nonce  ~--- целая величина, количество транзакций отправленных с этого адреса
\end{itemize}

nonce хранится для того, чтобы предотвратить двойную трату (double spending)\cite{double-spending}.

\noindent Транзакция содержит следующие данные:
\begin{itemize}
\item $s$ ~--- адрес отправителя
\item $r$ ~--- адрес получателя
\item $n$ ~--- nonce адрес отправителя в момент создания транзакции
\item $amount$ ~--- количество переводимых токенов
\item $pk$ ~--- публичный ключ отправителя (sender)
\item $sig$ ~--- подпись кортежа $(s, r, amount, n)$, сделанная с помощью секретного ключа $s$
\end{itemize}

\noindent Таким образом, каждый участник системы может проверить, что подпись действительно сделана с помощью секретного ключа, соответствующего публичному ключу pk. Попытка подделать кортеж $(s, r, amount, n)$, чтобы подпись осталась такой же, является NP полной задачей.

После того как участник системы получает транзакцию $t$, он проверяет, что выполняются следующие условия:
\begin{itemize}
\item $\boldsymbol{\sigma}[s].balance \ge t.amount$ ~--- отправитель имеет достаточное количество токенов
\item $\boldsymbol{\sigma}[s].nonce = t.n$ ~--- nonce транзакции совпадает с хранимым в хранилище
\item $verifySignature(t.sig, t.pk, (t.s, t.r, t.amount, t.n))$ ~--- подпись в транзакции корректна
\end{itemize}

Если все вышеописанные условия выполняются, то участник обновляет локальную копию хранилища следующим образом:

$\boldsymbol{\sigma}[s].balance = \boldsymbol{\sigma}[s].balance - t.amount$

$\boldsymbol{\sigma}[r].balance = \boldsymbol{\sigma}[r].balance + t.amount$

$\boldsymbol{\sigma}[s].nonce = \boldsymbol{\sigma}[s].nonce + 1$\\
Все данные изменения должны проводиться атомарно, чтобы избежать неконсистентного состояния хранилища.

%\subsection{Хранилище на основе списка непотраченых выходов}

%\section{Криптографические примитивы}

\section{Существующие криптовалюты и решения}

\subsection{Bitcoin и Bitcoin-NG}
Как уже было отмечено ранее, Bitcoin использует инклюзивный алгоритм консенсуса, который подробно описано в \cite{nakamoto}. Данный алгоритм состоит из следующих основных этапов.
Каждый участник поддерживает актуальные блокчейн и хранилище. Получая транзакции, участник формирует блок, в котором содержатся данные транзакции и ссылка на предыдущий блок. Далее участник, который называется \textit{майнер}, пытается найти некоторое целое значение $\phi$, которе будучи захэшированным вместе с хэшем содержимого созданного блока, даст значение хэш функции с определенным количеством нулей, которое называется \textit{цель} (target), подобно тому, как описано в работе \cite{hashcash}. Найдя данное значение, участник отправляет блок другим участникам, которые проверяют, что данное значение действительно удовлетваряет заданной цели, которая известна всем участникам, а также транзакции удовлетворяют условиям описанным в разделе \ref{sec:tx}. Для хэширования используется криптоустойчивая хэш функция, например SHA256 или SHA512 \cite{sha-2}.
Выпущенные блоки формируются в цепочку, которая является доказательством корректности транзакций. Алгоритмы консенсуса в основе которых лежит вычисление какой-то сложной функции для доказательства, называются  алгоритмы консенсуса на основе \textit{доказательства выполненной работы}\cite{pow}.

Цель меняется со временем, и зависит от скорости выпуска блоков. Она адаптируется таким образом, чтобы время нахождение нужного значения $\phi$  приблизительно равнялось 10 минутам. Вычисление цели подробно описано в оригинальной статье \cite{nakamoto} и рассмотрение этого вычисления не столь важно для данной работы, более важно то, что данная величина в 10 минут является ключевой, и не может быть уменьшена без потерь гарантий безопасности.

При возникновении \textit{вилки} (fork), то есть альтернативной цепочки, которая также являются продолжением блокчейна, участники выбирают наидлиннейшую корректную из цепочек. Однако, византийский участник может выбрать цепочку, которая не является корректной, в которой например содержатся транзакции, которые производят двойную трату. Данный алгоритм устойчив к византийским участникам, если их суммарная вычислительная мощность не превосходит $p$ \% вычислительной мощности всех участников системы. В оригинальной статье, $p$ заявлялось равным 50, однако дальнейшие исследования показали, что существуют атаки, которые нарушают корректность алгоритма даже при $p=25$ \cite{DBLP:journals/corr/EyalS13}.

Чтобы считаться транзакцию \textit{подтвержденной}, отправивший ее участник сети должен дождаться $O(\lambda)$  блоков после блока, в который она попала. Иначе, блок с данной транзакцией может быть \textit{откачен} (rollback), из-за возникшей вилки. Считается, что при $\lambda=6$ вероятность достаточна мала, чтобы считать отмену блоков возможной.\vspace{10pt}

Резюмируя все вышесказанное, Bitcoin имеет две основные проблемы:
\begin{enumerate}
\item большое время попадания транзакции в блок, которое равно 10 минутам. Как следствие ~--- маленькая пропускная способность;
\item время подтверждения транзакции равно 60 минутам.
\end{enumerate} \vspace{10pt}

Bitcoin-NG \cite{bitcoin-ng} является усовершенствованием Bitcoin и решает первую проблему. В предложенном в 2016 году алгоритме майнер, нашедший число $\phi$,  получает право создавать \textit{микроблоки}, в которых будут находиться транзакции. Выпуск микроблоков происходит намного чаще основных блоков ~--- раз в 10 секунд. Данное решение значительно увеличивает скорость обработки транзакций, а также уменьшает размер блоков.
Однако, предложенный алгоритм все еще страдает от второй проблемы ~--- время подтверждения все еще равно 60 минутам.

Слишком долгое время подтверждения транзакции ~--- это не только проблема Bitcoin, но и многих других криптовалют, которые предоставляют \textit{консистентность в конечном счете}, то есть в которых последние $O(\lambda)$ блоков могут быть откачены. Данная проблема присуща таким известным проектам как Etherium\cite{buterin2014ethereum} и Cardano \cite{cardano}.

Данная проблема была формализована в статье Hybrid Consensus \cite{hybrid-consensus} теоремой, следствие из которой звучит как: 
\par \textit{не существует отзывчивого протокола, который бы оставался безопасным при доле нечестных участников превышающей $\frac{1}{3}$ от общего числа}.

Отдельного внимания заслуживает термин \textit{отзвывчивость} протокола. Неформально он означает, что подтверждение транзакции ограничено некоторой величиной $O(\delta)$, где $\delta$ ~--- это время распространения транзакции по сети. Он будет введен формально в следующей главе.
 
Данное следствие распространяется как на экслюзивные, так и на инклюзивные протоколы. 
В частности, оно означает, что Bitcoin и подобные протоколы, не могут обрабатывать транзакции с временем подтверждения $O(\delta)$.

\subsection{Алгоритм XFT}
В работе\cite{DBLP:journals/corr/LiuCQV15}, предложенной  Shengyun Liu, описывается решение, которое основывается на предположении, что византийская ошибка ~--- это слишком <<сильная>> модель для злоумышленника. Авторы работы утверждают, что в реальности захватить все каналы связи довольно проблемантично, поэтому это допущение можно ослабить.

Основываясь на этих предположениях, авторы работы предлагают алгоритм, который справляется с $f$ византийскими участниками, с общим количеством участников равным $2f+1$. Среди всех участников выбирается кворум из $f+1$, которые синхронным образом достигают консенсуса о порядке операций.
Однако ключевым недостатком данной работы является то, что если среди кворума есть хотя бы один участник, который действует не согласно предписанному алгоритму, то кворум не сможет достигнуть консенсуса, и произойдет смена кворума. Более того, в работе делается сильное допущение, что все участники кворума могут коммуницировать между собой за время не большее некоторого $\Delta$, и если это не выполнено, то опять же произойдет смена кворума.

С учетом того, что кворум всегда выбирается произвольным образом, и при $t$ византийских участниках, вероятность выбрать кворум без хотя бы одного из них грубо можно оценить как $2^{-t}$, что неприемлимо, еще и с учетом того, что даже среди честных участников, некоторые соединения могут быть нестабильными.

\subsection{Алгоритмы ByzCoin и Solida}
Данная дипломная работа вдохновлена предложенной в 2016 году статье \cite{byzcoin}. В этой статье в одной из первых описывается идея, в которой комитет участников упорядочивает проводимые транзакции, и комитет обновляется со временем. В статье описывается подход, который справляется с $f$ византийскими участниками, при размере комитета $3f+1$. В основе подхода лежит алгоритм PBFT\cite{pbft}, и это и является первым недостатком этой работы: данный алгоритм требует $n^2$ сообщений, на один блок с транзакциями, однако в работе приведен подход, который уменьшает количество сообщения до $n$, но ухудшает безопасность. Количество сообщений может быть уменьшено с сохранением гарантий безопасности, и в данной дипломной работе это будет продемонстрировано.

Следующим недостатком работы, является то, что в ней крайне расплывчато описан процесс \textit{реконфигурации}, то есть каким образом обновляется состав комитета. Данный недостаток был частично исправлен в последующей работе, названной Solida\cite{solida}. Однако в Solida подход все еще уязвим к атакам. В данной дипломной работе предлагается подход, который устраняет присущие Solilda недостатки.
 
\finishrelatedwork

%-*-coding: utf-8-*-

\chapter{Описание предложенного алгоритма}
В данной главе будет описан предложенный алгоритм консенсуса на основе комитета участников, проведен его анализ и доказательство,
а также возможные модификации.

\section{Схема предложенного алгоритма}
В данном разделе будет описана основная схема и краеугольные камни алгоритма.

В основе алгоритма находится комитет участников сети, 
которые принимают решения по всем жизненно важным процесам криптовалюты, таким как: создание новых блоков с транзакциями, их верификацию и другие. 
Каждый участник может быть идентифицируем некоторым образом, например, его публичным ключом, и каждый из участников
знает индефикаторы всех остальных участников комитета.
Комитет состоит из $n$ участников сети, размер его не меняется с течением времени. 
Новый участник может присоединиться к нему, при этом один из существующих должен покинуть его, чтобы размер оставался равным $n$.

Комитет ответственнен за корректное состояние хранилища и порядка операций, изменяющих его.
Если какой-то из участников хочет внести некоторое изменение в хранилище, остальные участники должны прийти к консенсусу, является ли оно корректным. 
Выбор алгоритма, с помощью которого участники будут приходить к общему решению, является одной из важных частей решения.  
По сути, участники должны реализовывать SMR(State Machine Replication)[ссылка].

Таким образом, алгоритм условно можно разбить на две составляющие: алгоритм SMR в его основе и то, каким образом новые участники попадают в комитет.

TODO картинка

\section{Свойства SMR}
В данном разделе мы обозначим свойства, которые должен предоставлять SMR алгоритм.
В в следующих разделах рассмотрим возможных канидатов, удовлетворяющих этому свойству.

Первое из таких свойств \textit{консистентность}[ссылка на Hybrid consensus], включает в себя:
\begin{itemize}
\item Общий префикс - TODO
\item Самоконсистентность - TODO
\end{itemize}
%-*-coding: utf-8-*-

\chapter{Анализ предложенного алгоритма}  \label{chapter3}

\section{Модель и определения}
Прежде всего опишем модель, в которой будут проводиться следующие рассуждения, доказательства лемм и теорем.

Определим \textit{события} в системе. Событиями будем считать:
\begin{itemize}
\item отправку сообщения
\item получение сообщения
\item запуск таймера
\item остановку таймера
\end{itemize}

Будем считать, что событие отправки сообщения происходит до его получения, запуска таймера до его остановки, а также, что события в рамках одного участника могут быть линейно упорядоченны.

В итоге мы получили систему на основе событий с полным порядком. В данной системе можно ввести понятие \textit{согласованного среза}[ссылка]. Также введем в данной системе глобальное время. Каждому событию в системе можно сопоставить некоторый глобальный момент времени.

Прежде всего, каждый участник имеет состояние, в которое входят следующие переменные:
\begin{itemize}
\item leaderChange :: Bool
\end{itemize}

Обозначим состояние $i$-го участника в момент времени $t$ как $S_i(t)$, а состояние всей системы $S(t)$. 
$S(t)=\{S_1(t), S_2(t),..., S_{3f+1}(t)\}$. Также введем обозначение $variable^i(t)$ значение переменной $variable$ у участника $i$ в момент времени $t$.
Обозначим состояние $i$-го участника в согласованном срезе $C$ как $S_i(C)$, аналогично $S(C)$ и $variable^i$.
Если $S^i$, $S$ и $variable^i(t)$ используется без указания времени или среза, то предполагается, что $t$ или $C$ ясны из контекста, либо для любых $t$ и $C$.

Будем считать, что честные участники изменяют их состояния согласно описанному алгоритму, в то время как нечестные могут изменять произвольно.

Злые участники \textbf{TODO}.

\textit{Определение.} Будем называть изменение $d$ \textit{оседлым} в слоте $s$, если у как минимум $f+1$ честного участника $acceptance_s$~--- это корректный сертификат согласия для пары $(s, d)$.

\textit{Определение.} Будем считать, что честный участник $i$ находится в \textit{состоянии смены лидера}, если $leaderChange^i = True$.

\textit{Определение.} Будем считать, что честный участник $i$ находится в \textit{устойчивом состоянии}, если $leaderChange^i = False$.

\textit{Определение.} Состояние системы $S$ называется \textit{устойчивым состоянием}, если не более $f$ честных участников находится в состоянии смены лидера.

Если система или участник не находится в устойчивом состоянии, будем говорить, что они находится в \textit{неустойчивом} состоянии.

\section{Базовые свойства}
В данном разделе будут доказаны некоторые базовые свойства, который будут использоваться в дальнейшем.

\textbf{\textit{Свойство 1.}} Весь жизненный цикл системы разбивается на последовательность времен
$t_0 \le T_1 \le T_2 \le T_3 \le ...$, $T_0=0$, где в полуитервалы времени $[T_{2k}...T_{2k+1})$ система находится в устойчивом состоянии, а в полуинтервалы $[T_{2k+1}...T_{2k+2})$ в неустойчивом.

\textbf{\textit{Лемма 1.}} \textit{В каждом устойчивом состоянии существует такой согласованный срез $C$, для которого выполняется, что у всех честных участников, кроме может быть $f$ из них, $s^i$ равны некоторому значению $s$ и $acceptance^i_s$ одинаковы.}

Это очевидно верно для устойчивого состояния $[T_0...T_1)$, срез $C$ можно взять по $T_0$.

Рассмотрим некоторое устойчивое состояние, а также предыдущее ему состояние смены лидера.
Если хотя бы у одного честного участника $i$ значение $leaderChange$ стало $True$, это значит, что он получил сам и отправил другим $\mathcal{W}$. Получая $\mathcal{W}$, участник актуализирует лог вплоть до $s^{*}$ и присваивает $acceptance_{s_P^{*}} := \hat{\mathcal{P}}^{*}$. Если в какой-то момент времени, состояние стало устойчивым, значит у не более чем $f$ честных участников $leaderChange = False$, следовательно, для остальных существовал такой момент времени $t_i$, в который $s^i$ был равен $s^{*}$ и $acceptance_{s_P^{*}}$ равен $\hat{\mathcal{P}}^{*}$. Возьмем эти моменты времени $t_i$ в качестве среза $C$. $\square$

\section{Консистентность алгоритма}
В данном разделе будет дано определение \textit{консистентности} (Consistency)\cite{hybrid-consensus} и доказано, что предложенный алгоритм обладает этим свойством.

Будем считать, что алгоритм обладает свойством \textit{общего префикса}, если в любой момент для любых двух честных участников комитета (возможно одного и того же) $i$ и $j$ выполняется: либо $Log_{C_i}$ префикс $Log_{C_j}$, либо $Log_{C_j}$ префикс $Log_{C_i}$. Считаем, что $x$ префиксом самого себя и $\emptyset$ префиксом $x$.

Будем считать, что честный участник удовлетворяет свойству \textit{самоконсистентности}, если выполняется: пусть участник честный во время $t$ и $t' \ge t$, $Log_{C_i}$~--- лог в момент времени $t$, $Log'_{C_i}$~--- в момент времени $t'$, тогда верно, что $Log_{C_i}$ префикс $Log'_{C_i}$.

Будем называть алгоритм \textit{консистентным}, если для него выполняется свойство общего префикса и для любого участника выполняется свойство самоконсистентности.
 
Далее будет приведено доказательство консистентности предложенного алгоритма.

\textbf{\textit{Лемма 1.}} \textit{В любой момент времени для любого слота $s$ существует не более одного оседлого изменения $d$.}

Предположим, что существуют два оседлых изменения $d_1$ и $d_2$ в слоте $s$. Рассмотрим сертификат согласия $\mathcal{P}_1$ любого участника для $d_1$ и сертификат согласия $\mathcal{P}_2$ любого участника для $d_2$. Обозначим за $(c_1, v_1)$ значения $c$ и $v$ из первого сертификата, и $(c_2, v_2)$ из второго.

Обозначим за $A$ множество участников отправивших$Prepare$ сообщения для  $\mathcal{P}_1$ ($|A|=2f+1$), за $B$ ($|B|=2f+1$) сообщения для $\mathcal{P}_2$. $|A \cap B| \ge f+1$, отсюда следует, что в пересечении найдется как минимум один честный участник,  обозначим его за $x$. Получаем, что $x$ отправил два $Prepare$ сообщения с разными $d_1$ и $d_2$: одно с $(c_1, v_1, s)$ и второе с $(c_2, v_2, s)$.

Допустим, что $c_1 = c_2$ и  $v_1 = v_2$. Тогда получаем противоречие с проверками 1.1 из раздела \ref{steady-state}.

Допустим теперь, что $c_1 < c_2$. Это бы значило, что $x$ между отправками $Prepare$ сообщений должен был бы перейти в другую конфигурацию, сделать он это может только добавив в свой лог $\mathcal{C}$ с $d_{rec}$, тем самым увеличив свое значение слота, но тогда бы он не смог отправить второе $Prepare$ сообщение для того же самого $s$, что и первое. Аналогично для случая $c_2 < c_1$.

Допустим теперь, что $c_1=c_2$ и $v_1 < v_2$. Так как $v_1 < v_2$, то должна была произойти смена лидера и переход в новое устойчивое состояние, в котором не более $f$ честных участников, у которых текущий слот $s$, и у остальных участников текущий слот больше $s$. Отсюда следует, что в сертификат для  $\mathcal{P}_2$ не может содержать $2f+1$ сообщений со слотом $s$, так как это противоречит проверкам 1.1 из раздела \ref{steady-state}.  Получили невозможность существования $\mathcal{P}_2$~--- противоречие. Аналогично для случая $v_2 < v_1$.
$\square$

\textbf{\textit{Лемма 2.}} Если в некоторый момент времени для слота $s$ существует некоторое оседлое изменение $d$, то данное изменение в конечном счете неизбежно окажется в логе применения всех честных участников в слоте $s$.

Ра
TODO $\square$

\vspace{10pt}

\textit{Следствие из Леммы 1.} Если хотя бы один честный участник имеет сертификат применения $\mathcal{C}$ для изменения $d$ соответствующему слоту $s$, то  изменение $d$ неизбежно окажется в логе применения всех честных участников в слоте $s$. 

Так как, из того что участник имеет $\mathcal{C}$, следует что как минимум $f+1$ честных участников отправили $Commit$ сообщение, имея сертификат согласия для $d$ и $s$. $\square$
\vspace{10pt}

\textbf{Теорема.} \textit{В любой момент времени логи честных участников консистентны.}

Предположим, что в некоторый момент времени существуют различные участники $i$ и $j$, такие что, существует $s$, для которого выполняется, что изменение из $Log_{C_i}[s]$ неравно изменению из $Log_{C_j}[s]$. Тогда существовал некоторый момент времени, в который либо изменение $Log_{C_i}[s]$, либо изменение $Log_{C_j}[s]$ было оседлым, отсюда по Лемме 2 получаем, что только это изменение могло оказаться в слоте $s$ у всех честных участников. Получили противоречие.

Свойство самоконсистентности очевидно, так как в описанном алгоритме изменения не удаляются из лога применений и не модифицируются, а только добавляются в него. $\square$

\section{Прогресс алгоритма}
\noindent Следующее свойство ~--- это свойство \textit{прогресса} (Liveness) \cite{hybrid-consensus}:
пусть честный участник получил транзакцию во время $t$, тогда данная транзакция будет добавлена в хранилище всеми \textit{честными} участниками не позднее $t + T_{confirm}$.

Данное свойство использует параметр $T_{confirm}$, предполагается что данный параметр постоянен на протяжении всего времени и известен зараннее.

\section{Отзывчивость алгоритма}

\startconclusionpage

В рамках данной работы был предложен алгоритм консенсуса для распределенных геореплицируемых систем.
Два основных преимущества предложенного решения по сравнению с существующими состоят в том, что в алгоритме не может быть вилок, таким образом он обладает строгой консистентностью, второе, но не менее важное преимущество~--- уменьшены затраты на коммуникацию между участниками системы.

Алгоритм был рассмотрен в контексте разработки криптовалют, подробно описан в главе 2, а также проанализирован в главе 3. В результате анализа было доказано, что алгоритм обладает консистентностью и обеспечивает прогресс системы, также были приведены оценки на ожидаемое время обработки транзакций. Тем самым были продемонстрировано, что алгоритм предоставляет необходимые гарантии и показана жизнеспособность.

Основным недостатком предложенного решения является то, что в его основе лежит подход на основе доказательства проделанной работы, который требует использования большого количество энергии. Другим же недостатком является требование на нахождение в сети как минимум $2f+1$ активных участников. Данные проблемы могут быть решены, используя другие подходы, такие как доказательство доли владения и другие, которые лишены данных проблем, хотя и обладают другими.

В заключении, хочется отметить, что предложенный в данной работе подход~--- это еще один шаг в сторону более устойчивых и быстрых алгоритмов консенсуса. Алгоритм не лишен недостатков, но многие из них обозначены в рамках работы и могут быть устранены во время дальнейшей его разработки.

\printmainbibliography

%\appendix

%\input{appendicies/adagrad_1.tex}
%\input{appendicies/adagrad_2.tex}
%\input{appendicies/adam_1.tex}
%\input{appendicies/adam_2.tex}
\end{document}
